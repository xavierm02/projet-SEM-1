\documentclass[11pt]{scrartcl}

%% Corrige quelques erreurs de LaTeX2ε
\usepackage{fixltx2e}
\usepackage{xspace}
\usepackage{microtype}

%% Pour ne pas commettre les erreurs fréquentes décrites dans l2tabu
\usepackage[l2tabu,abort]{nag}

%% Saisie en UTF-8
\usepackage[utf8]{inputenc}
%% Fonte : cf. http://www.tug.dk/FontCatalogue/
\usepackage[T1]{fontenc}
\usepackage{lmodern}

%% Pour écrire du français
\usepackage[frenchb]{babel}

%% Pour composer des mathématiques
\usepackage{mathtools}

%% Quelques paquets qui peuvent se révéler utiles pour parler
%%  des langages de programmation et de leurs sémantiques.
%% Consultez-en la documentation.
%\usepackage{listings}
%\usepackage{semantic}

% pour l’exemple…
\usepackage{kantlipsum}

\begin{document}

% Pour générer un titre
\author{Un~\bsc{Élève} \and Son~\bsc{Acolyte}}
\date{11 octobre 1582}
\title{De Semantica Rerum}
\maketitle

\begin{abstract}
  \kant[1]
\end{abstract}

\section{Introduction}
\kant[2-3]

\section{Titre de la section~\ref{sec:labsec1}}
\label{sec:labsec1}
\subsection{Une sous-section}
\kant[4-5]
\subsection{Une autre sous-section}
\kant[5-6]

\section{Titre de la section~\ref{sec:labsec2}}
\label{sec:labsec2}
\kant[7-8]

\section{Titre de la section~\ref{sec:labsec3}}
\label{sec:labsec3}
\kant[9-10]

\section{Conclusion}
\kant[164]

\end{document}
